\section{Introduction}%
\label{sec:introduction}

\subsection{Background}%
\label{sub:background}
% See project plan

\subsection{Purpose of the project}%
\label{sub:purpose_of_the_project}
The project aims to show that a reliable homebrew fermentation temperature logging system, with data visualization, can be created using open source software and easily accessible electronics. A list of used hardware and source code of the finished product will be published to GitHub.

\subsection{Work distribution and planning}%
\label{sub:work_distribution_and_planning}
% GANTT
\begin{figure}[h]
  \centering
  \begin{ganttchart}[y unit title=0.4cm,
  y unit chart=0.5cm,
  x unit=0.3cm,
  vgrid,hgrid, 
  title label anchor/.style={below=-1.6ex},
  title left shift=.05,
  title right shift=-.05,
  title height=1,
  progress label text={},
  bar height=0.7,
  group right shift=0,
  group top shift=.6,
  group height=.3]{1}{40}

  %Labels
  \gantttitle{Project Timeline}{40} \\
  \gantttitle{Phase 1}{15}
  \gantttitle{Phase 2}{10}
  \gantttitle{Final Phase}{15} \\
  \gantttitle{Week 1}{5} 
  \gantttitle{Week 2}{5} 
  \gantttitle{Week 3}{5} 
  \gantttitle{Week 4}{5} 
  \gantttitle{Week 5}{5} 
  \gantttitle{Week 6}{5} 
  \gantttitle{Week 7}{5} 
  \gantttitle{Week 8}{5} \\

  %Tasks
  \ganttbar[progress=0]{Task 1}{1}{5} \\
  \ganttlinkedbar[progress=0]{Task 2}{6}{15} \\
  \ganttlinkedbar[progress=0]{Task 3}{16}{25} \\
  \ganttlinkedbar[progress=0]{Task 4}{26}{35} \\
  \ganttlinkedbar[progress=0]{Task 5}{36}{40}

\end{ganttchart}

  \caption{Gantt scheme showing the enumerated tasks during the project timespan}
  \label{fig:gantt}
\end{figure}

The work planning and execution is done by one person with the help of a supervisor, which is also head of the course. As seen in Fig, the duration of the project was split into three phases with the enumerated tasks

The time planning for the enumerated tasks is visualized as a Gantt scheme in Figure \ref{fig:gantt}.
\begin{enumerate}
  \item
Set up project repository, as well as report and documentation workflow. Formulate preliminary time plan and goals. Research suitable hardware.
  \item
Get started with the AVR toolchain, write Makefiles and configure a development environment. Configure ARM compatible Linux on the RPi. Write UART routines and verify that the communication is working. Extend the communication to contain temperature data from the sensor.
  \item
    Apply a filter to the signal. Design the website, configure databases and visualization of data.
  \item
    Extend the website with more features. Collect or generate dummy-data in order to present a proof of concept.
  \item Present the project, i.e all project phases, and finalize the report.
\end{enumerate}

\subsection{Grading criteria}%
\label{sub:grading_criteria}
The finally revised grading criteria is
\begin{itemize}
  \item 
    Grade 3. The temperature is measured by the microcontroller and presented through a visualization on a website that is accessible from the internet. The website is hosted on a Raspberry Pi computer configured with an appropriate ARM compatible Linux distribution.
  \item 
   Grade 4. The temperature sensor signal can be processed through a filter. If a drastic temperature is detected, an automated script sends an alert through email. 
  \item
   Grade 5. In addition to the stated grading criteria, the data is accessed in a manner similar to an interactive process view that a production company might order as a web application. The data collection is reliable and handles unexpected errors that might affect the system.
\end{itemize}

The goals and specifications of the project is set during the beginning of the first phase but can be reformulated according to the progress. Changes to the specifications and/or grading criteria is discussed and decided by the project supervisor.
