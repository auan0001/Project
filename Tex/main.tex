\documentclass[10pt]{article}
% -------------------------------------------------------------------
\usepackage[utf8]{inputenc}				
\usepackage[T1]{fontenc}					
\usepackage[english]{babel}
\usepackage[margin=5cm]{geometry}
% -------------------------------------------------------------------
\usepackage{pgfgantt}
\usepackage{xspace}
\usepackage{graphicx}
\usepackage{rotating}
\usepackage{amsmath,amssymb}
\usepackage{subfig,epsfig,tikz,float}		            % Packages de figuras. 
% -------------------------------------------------------------------
\usepackage{booktabs,multicol,multirow,tabularx,array}          % Packages para tabela
% -------------------------------------------------------------------
\setlength{\parindent}{0pt}
\setlength{\parskip}{5pt}
\textwidth 13.5cm
\textheight 19.5cm
\columnsep .5cm
% -------------------------------------------------------------------
\newcommand{\AVR}{\textsc{Avr}\xspace}
\title{\renewcommand{\baselinestretch}{1.17}\normalsize\bf%
  Project in Embedded Systems 15hp 1TE721\\
  \vspace{2mm}
  \uppercase{A fermentation temperature monitoring system}\\
  using Atmel AVR and ARM single-board computer\\
}
% -------------------------------------------------------------------
\author{%
  August Forsman\\
  \small aufo8456@student.uu.se
}
% -------------------------------------------------------------------
\begin{document}

\date{}

\maketitle

\vspace{-0.5cm}

%-------------------------------------------------------------------
%Abstract
%\bigskip
%\noindent
%{\small{\bf ABSTRACT.}
%Abstract should concisely
%summarize the key findings of the paper. It should consist 
%of a single paragraph containing no more than 150 words. 
%The Abstract does not have a section* number.
%}

%\medskip
%\noindent
%{\small{\bf Keywords}{:} 
%After the abstract three keywords must be provided.
%}

\baselineskip=\normalbaselineskip
% -------------------------------------------------------------------

\section*{Background}%
\label{sec:background}
In recent years, the number of small scale breweries has skyrocketed in numbers. This has led to an increased demand of advanced brewing systems operating on much smaller volumes compared to the macro-scale brewing industry. These systems are often equipped with sensors logging temperature and pressure.

A traditional brewing style, often referred to as 
\textit{farmhouse brewing}, focuses on letting a mixed flora of yeasts and bacteria (commonly Brettanomyces, Pediococcus, Lactobacillus) free-rise in temperature, hence not using any control systems. Since the compounds produced by these microorganisms are in most cases unwanted, a lot of research has been put into detecting an excluding them from the process as part of quality control. An important step in getting to know more of the brewer's house culture is logging the fermentation temperature, which in some cases can be as long as three years.


\section*{Objectives}%
\label{sec:objectives}
The objective of this project is develop a small and portable system prototype that allows an AVR MCU to communicate with a Raspberry Pi (RPi) through serial communication. The signal should not contain any unnecessary noise and can therefor be filtered by the MCU before transmitting to the server.

The RPi based server is accessed on the internet by the end user and can be visualized using external libraries that works on both computers and mobile units.
\newpage
\section*{Preliminary tasks}%
\label{sec:preliminary_tasks}
The time planning for the enumerated tasks is visualized as a Gantt scheme in Figure \ref{fig:gantt}.
\begin{enumerate}
  \item
Setup project repository, as well as report and documentation workflow. Formulate preliminary time plan and goals. Look for suitable hardware.
  \item
Get started with the \AVR toolchain, write Makefiles and configure a development environment. Configure ARM compatible Linux on the RPi. Write UART routines and verify that the communication is working. Extend the communication to contain temperature data from the sensor.
  \item
    Apply a filter to the signal. Design the website, configure databases and visualization of data.
  \item
    Extend the website with more features. Collect or generate dummy-data in order to present a proof of concept.
  \item Present the project, i.e all project phases, and finalize the report
\end{enumerate}
\section*{Grading criteria and grade to be achieved\protect\footnote{Suggestions. To be discussed.}}%
\label{sec:grading_criteria_and_grading_goal}
\begin{itemize}
  \item 
    Grade 3. The temperature is measured by the microcontroller and presented through a visualization on a website that is accessible from the internet. The website is hosted on a Raspberry Pi computer configured with an appropriate ARM compatible Linux distribution.
  \item 
   Grade 4. The thermostat signal is processed through a Kalman filter or similar. After the initial fermentation, the ambient conditions governs the temperature of the beer. The code is portable and it is possible to extend the system with temperature control in a fermentation chamber.
  \item
   Grade 5. In addition to the stated grading criteria, the data is accessed in a manner similar to an interactive process view that a production company might order as a web application.
\end{itemize}

\newpage
\section*{Preliminary time schedule}%
\label{sec:preliminary_time_schedule}


\begin{figure}[ht]
  \centering
  \begin{ganttchart}[y unit title=0.4cm,
  y unit chart=0.5cm,
  x unit=0.3cm,
  vgrid,hgrid, 
  title label anchor/.style={below=-1.6ex},
  title left shift=.05,
  title right shift=-.05,
  title height=1,
  progress label text={},
  bar height=0.7,
  group right shift=0,
  group top shift=.6,
  group height=.3]{1}{40}

  %Labels
  \gantttitle{Project Timeline}{40} \\
  \gantttitle{Phase 1}{15}
  \gantttitle{Phase 2}{10}
  \gantttitle{Final Phase}{15} \\
  \gantttitle{Week 1}{5} 
  \gantttitle{Week 2}{5} 
  \gantttitle{Week 3}{5} 
  \gantttitle{Week 4}{5} 
  \gantttitle{Week 5}{5} 
  \gantttitle{Week 6}{5} 
  \gantttitle{Week 7}{5} 
  \gantttitle{Week 8}{5} \\

  %Tasks
  \ganttbar[progress=0]{Task 1}{1}{5} \\
  \ganttlinkedbar[progress=0]{Task 2}{6}{15} \\
  \ganttlinkedbar[progress=0]{Task 3}{16}{25} \\
  \ganttlinkedbar[progress=0]{Task 4}{26}{35} \\
  \ganttlinkedbar[progress=0]{Task 5}{36}{40}

\end{ganttchart}

  \caption{Gantt scheme representing the project timeline.}
  \label{fig:gantt}
\end{figure}
\end{document}
