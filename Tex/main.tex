\documentclass[10pt]{article}
% -------------------------------------------------------------------
\usepackage[utf8]{inputenc}				
\usepackage[T1]{fontenc}					
\usepackage[english]{babel}
\usepackage[margin=5cm]{geometry}
% -------------------------------------------------------------------
\usepackage{pgfgantt}
\usepackage{xspace}
\usepackage{graphicx}
\usepackage{rotating}
\usepackage{amsmath,amssymb}
\usepackage{subfig,epsfig,tikz,float}		            % Packages de figuras. 
% -------------------------------------------------------------------
\usepackage{booktabs,multicol,multirow,tabularx,array}          % Packages para tabela
% -------------------------------------------------------------------
\setlength{\parindent}{0pt}
\setlength{\parskip}{5pt}
\textwidth 13.5cm
\textheight 19.5cm
\columnsep .5cm
% -------------------------------------------------------------------
\newcommand{\AVR}{\textsc{Avr}\xspace}
\title{\renewcommand{\baselinestretch}{1.17}\normalsize\bf%
  Project in Embedded Systems 15hp 1TE721\\
  \vspace{2mm}
  \uppercase{Fermentation temperature monitoring system}\\
  using Atmel AVR and Raspberry Pi\\
}
% -------------------------------------------------------------------
\author{%
  August Forsman\\
  \small aufo8456@student.uu.se
}
% -------------------------------------------------------------------
\begin{document}

\date{}

\maketitle

\vspace{-0.5cm}

%-------------------------------------------------------------------
%Abstract
%\bigskip
%\noindent
%{\small{\bf ABSTRACT.}
%Abstract should concisely
%summarize the key findings of the paper. It should consist 
%of a single paragraph containing no more than 150 words. 
%The Abstract does not have a section* number.
%}

%\medskip
%\noindent
%{\small{\bf Keywords}{:} 
%After the abstract three keywords must be provided.
%}

\baselineskip=\normalbaselineskip
% -------------------------------------------------------------------

\section*{Background}%
\label{sec:background}
\section*{Preliminary tasks}%
\label{sec:preliminary_tasks}
\section*{Grading criteria and grade to be achieved\protect\footnote{Suggestions. To be discussed.}}%
\label{sec:grading_criteria_and_grading_goal}
\subsection*{Grade 3}%
\label{sub:grade_3}
The temperature is measured by the microcontroller and presented through a visualization on a website that is accessible from the internet. The website is hosted on a Raspberry Pi computer configured with an appropriate ARM compatible Linux distribution.
\subsection*{Grade 4}%
\label{sub:grade_4}
The thermostat signal is processed through a Kalman filter or similar. After the initial fermentation, the ambient conditions governs the temperature of the beer. The code is portable and it is possible to extend the system with temperature control in a fermentation chamber.
\subsection*{Grade 5}%
\label{sub:grade_5}
In addition to the stated grading criteria, the data is accessed in a manner similar to an interactive process view that a production company might order as a web application.

Grade to be achieved: 5
\section*{Preliminary time schedule}%
\label{sec:preliminary_time_schedule}
\subsection*{Task 1}%
\label{sub:task_1}
Setup project repository, as well as report and documentation workflow. Formulate preliminary time plan and goals. Look for suitable hardware.
\subsection*{Task 2}%
\label{sub:task_2}
Get started with the \AVR toolchain, write makefiles and configure .


\begin{figure}[ht]
  \centering
  \begin{ganttchart}[y unit title=0.4cm,
  y unit chart=0.5cm,
  x unit=0.3cm,
  vgrid,hgrid, 
  title label anchor/.style={below=-1.6ex},
  title left shift=.05,
  title right shift=-.05,
  title height=1,
  progress label text={},
  bar height=0.7,
  group right shift=0,
  group top shift=.6,
  group height=.3]{1}{40}

  %Labels
  \gantttitle{Project Timeline}{40} \\
  \gantttitle{Phase 1}{15}
  \gantttitle{Phase 2}{10}
  \gantttitle{Final Phase}{15} \\
  \gantttitle{Week 1}{5} 
  \gantttitle{Week 2}{5} 
  \gantttitle{Week 3}{5} 
  \gantttitle{Week 4}{5} 
  \gantttitle{Week 5}{5} 
  \gantttitle{Week 6}{5} 
  \gantttitle{Week 7}{5} 
  \gantttitle{Week 8}{5} \\

  %Tasks
  \ganttbar[progress=0]{Task 1}{1}{5} \\
  \ganttlinkedbar[progress=0]{Task 2}{6}{15} \\
  \ganttlinkedbar[progress=0]{Task 3}{16}{25} \\
  \ganttlinkedbar[progress=0]{Task 4}{26}{35} \\
  \ganttlinkedbar[progress=0]{Task 5}{36}{40}

\end{ganttchart}

  \caption{Gantt scheme representing the project timeline.}
  \label{fig:}
\end{figure}
\end{document}
