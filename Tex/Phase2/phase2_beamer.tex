% Uppsala University presentation template
% Created 2021-03 by Mats Jonsson, mats@mekeriet.se

\documentclass[aspectratio=169, 9pt]{uu-beamer}
\usepackage{mwe} % for example image
\usepackage{listings}
\lstset{
  belowcaptionskip=0em,
  aboveskip = 1em,
  belowskip=2em,
  framexbottommargin=0pt,
  language=bash,                % choose the language of the code
  basicstyle=\ttfamily\footnotesize,
  numbers=left,                   % where to put the line-numbers
  stepnumber=1,                   % the step between two line-numbers.
  numbersep=5pt,                  % how far the line-numbers are from the code
  backgroundcolor=\color{white},  % choose the background color. 
  showspaces=false,               % show spaces adding particular underscores
  showstringspaces=false,         % underline spaces within strings
  showtabs=false,                 % show tabs within strings adding particular underscores
  tabsize=2,                      % sets default tabsize to 2 spaces
  captionpos=b,                   % sets the caption-position to bottom
  title=\lstname,                 % show the filename of files included with \lstinputlisting;
frame=single,
    showstringspaces=false,%without this there will be a symbol in the places where there is a space
    numbers=left,%
    numberstyle={\tiny \color{black}},% size of the numbers
    numbersep=9pt, % this defines how far the numbers are from the text
    emph=[2]{word1,word2}, emphstyle=[2]{style},
    commentstyle=\color{blue},
    stringstyle=\color{purple},
}
\lstset{basicstyle=\ttfamily\footnotesize}
\lstset{extendedchars=\true}
\lstset{inputencoding=ansinew}
\lstset{breaklines=true}

\graphicspath{{images/}, {style/graphics/}} % Put graphics and images in images/ directory

% ------------------- Title -------------------------
\title{Phase 2 Presentation}
\subtitle{Project in Embedded Systems | 1TE721}
\author[A. Forsman]{August Forsman}
\date{\today}
\institute[UU]{Uppsala Universitet}

\begin{document}

% Logo page
%\logopage

% Main presentation title page
\titlepage


% Title page
%\anothertitle{Phase 2 Presentation}{Group 1 -- August Forsman}

% Two-column with image 
\begin{frame}
    \Frametitle{Summary of Phase 2}
    The work is finished according to the project plan.
    \vspace{5mm}
    \begin{columns}
        \column{0.7\textwidth}
        \begin{enumerate}
          \item Tuning the UART communication between the ATmega328p MCU and Raspberry Pi
          \item Choosing and configuring a suitable database and entry formatting
          \item Fail safe data collection  
          \item Testing the web application visualization with live data
        \end{enumerate}

        %\column{0.5\textwidth}
        %\includegraphics[keepaspectratio,width=\textwidth,height=\textheight]{example-image}
      \end{columns}
    
\end{frame}

% Two-column with image 
\begin{frame}
    \Frametitle{Data collection and visualization}
    A selection of used Python modules\ldots \\
    Down the dependency hole, we go!
    \begin{columns}
        \column{0.6\textwidth}
        \begin{enumerate}
          \item PySerial -- serial communication through UART 
          \item pyMongo -- setup and interact with MongoDB
          \item Plotly and Dash -- graphs and web app  
          \item Testing the web application visualization with live data
        \end{enumerate}

        \column{0.5\textwidth}
        \includegraphics[keepaspectratio,width=\textwidth,height=0.5\textheight]{Python_logo_icon.png}
      \end{columns}
    
\end{frame}

% Title, image and subtitle
\titleimage{Testing setup}{5 litre carboy with water by the window}{setup.jpg}

\begin{frame}[fragile]
    \Frametitle{Fetching the data}
    \begin{lstlisting}[caption={MCU program loop}, label={main}]
void main(void)
{
  /* Variables, external interrupts, UART init etc */
  for(;;)
  {
    /* Polling for input */
    if (uart_read_count() > 0)
    {
      data = uart_read();
      if (data == 'R') {
        ow_reset();
        ow_temp_rd(buffer);
        uart_send_arr(buffer, len);
        uart_send_byte('\r');
        uart_send_byte('\n');
      }
    }
  }
}
    \end{lstlisting}
\end{frame}

% Title and text
\begin{frame}[fragile]
  \Frametitle{Daemonizing data collection}
\begin{lstlisting}[caption={The systemd service that manages data collection}, label={systemd}]
[Unit]
# Human readable unit name
Description=Reads serially from '/dev/ttyUSB*' and puts in MongoDB

[Service]
# Command that executes script
ExecStart=/usr/bin/python /home/alarm/Project/serial_temp_to_db.py
# Redirect print() to the Linux journal
Environment=PYTHONUNBUFFERED=1
# Able to notify that the service is ready
Type=notify
Restart=always

[Install]
# Start service at boot
WantedBy=default.target
\end{lstlisting}
\end{frame}

\begin{frame}[fragile]
    \Frametitle{Accessing the journal}
    \begin{lstlisting}[caption={Init tests logged in the journal}, label={journal}]
~ % journalctl --user-unit=serial-temp-to-db.service
...
    -- Boot 492caa4c348b44e599a8070b1f5740d6 --
Feb 21 11:49:12 alarm systemd[361]: Starting Reads serially from /dev/ttyUSB* and puts in MongoDB...
Feb 21 11:49:59 alarm python[368]: Running tests on port: /dev/ttyUSB0
Feb 21 11:49:59 alarm python[368]: Baud rate = 9600
Feb 21 11:50:00 alarm python[368]: Faulty reading: 85.0000
Feb 21 11:50:00 alarm python[368]: Faulty reading: 85.0000
Feb 21 11:50:01 alarm python[368]: Correct reading: 19.1875
Feb 21 11:50:01 alarm python[368]: Correct reading: 19.1875
Feb 21 11:50:02 alarm python[368]: Correct reading: 19.1875
Feb 21 11:50:02 alarm python[368]: Passed init tests on 3 correct readings and 2 faulty readings
Feb 21 11:50:02 alarm systemd[361]: Started Reads serially from /dev/ttyUSB* and puts in MongoDB.
    \end{lstlisting}
\end{frame}

\begin{frame}[fragile]
    \Frametitle{Serial data to DB}
    \begin{lstlisting}[caption={The Python script running from systemd}, label={sertodb}]
client = MongoClient('localhost', IP_ADR) # MongoDB
...
ser = serial.Serial(PORT, BAUD)          # Open serial conn to MCU
...
def get_temperature(temp):
    ser.write(bytes(READ_CMD))           # Write read command
    line = ser.readline().decode()       # Read returning data
    if no errors and < MAX_FERM_TEMP:    # Error checking
        temp = (float(line.strip('\x00\n\r')))
    return temp

def temp_to_db(temp):                    # Dict DB entry
    temp_entry = {"temperature": get_temperature(temp),
            "time": dt.utcnow()}
    entry.insert_one(temp_entry)
    \end{lstlisting}
\end{frame}

\begin{frame}[fragile]
    \Frametitle{DB to graph}
    \begin{lstlisting}[caption={Getting data from DB and plotting using Plotly}, label={dbplot}]
client = MongoClient('localhost', IP_ADR) # MongoDB
db = client.beertemp

...

df = pd.DataFrame(list(db.entries.find().limit(int(LIVE_RES)).sort([('$natural',-1)]))) # Data in span from past to present
    trace = go.Scatter( # Scatter plot
        x=df['time'],
        y=df['temperature'])
...

return fig
    \end{lstlisting}
\end{frame}


% Full-size image
%\fullimage{example-image}
\begin{frame}
  \Frametitle{Live Demo}
\end{frame}

% Title and text
\begin{frame}
  \Frametitle{Conclusion}
  \begin{enumerate}
    \item IT WORKS!
    \item Timing issues and trash input solved
    \item I had to read a lot in order to decide \textit{what} to choose
    \item The workflow is very streamlined due to (magic) Python module integration
  \end{enumerate}
\end{frame}
% Title and text
\begin{frame}
  \Frametitle{Further work and the Final Phase}
  \begin{enumerate}
    \item Research different methods of deploying the web app 
    \item Design the application layout. More user interaction
    \item Working with the higher grade specifications
    \item Write the final report
  \end{enumerate}
  
\end{frame}

\end{document}

