\documentclass[10pt]{article}
% -------------------------------------------------------------------
\usepackage[utf8]{inputenc}				
\usepackage[T1]{fontenc}					
\usepackage[english]{babel}
\usepackage[margin=5cm]{geometry}
% -------------------------------------------------------------------
\usepackage[
    backend=biber,
    style=ieee,
    sortlocale=se_EN,
    natbib=true,
    url=true, 
    urldate=comp, 
    doi=true,
    eprint=false
]{biblatex}
\addbibresource{~/bibliography.bib}
\usepackage{xspace}
\usepackage{csquotes}
\usepackage{graphicx}
\usepackage{rotating}
\usepackage{amsmath,amssymb}
\usepackage{subfig,epsfig,tikz,float}		            % Packages de figuras. 
% -------------------------------------------------------------------
\usepackage{booktabs,multicol,multirow,tabularx,array}          % Packages para tabela
% -------------------------------------------------------------------
\setlength{\parindent}{0pt}
\setlength{\parskip}{5pt}
\textwidth 13.5cm
\textheight 19.5cm
\columnsep .5cm
% -------------------------------------------------------------------
\newcommand{\AVR}{\textsc{Avr}\xspace}
\title{\renewcommand{\baselinestretch}{1.17}\normalsize\bf%
  Phase 1 Report | Project in Embedded Systems 15hp 1TE721\\
  \vspace{2mm}
  \uppercase{A fermentation temperature monitoring system}\\
  using Atmel AVR and an ARM single-board computer\\
}
% -------------------------------------------------------------------
\author{%
  Group 1: August Forsman\\
  \small aufo8456@student.uu.se
}
% -------------------------------------------------------------------
\begin{document}

\date{}

\maketitle

\vspace{-0.5cm}

\baselineskip=\normalbaselineskip
% -------------------------------------------------------------------

\section*{Summary}%
Phase 1 of the project has mainly consisted of
\begin{enumerate}
  \item Configuring UART drivers for the ATmega328p MCU
  \item Setting up 1-Wire communication between the MCU and DS18B20
  \item Confirming that the communication works through serial communication between the MCU and Raspberry Pi
\end{enumerate}
The work is finished in time according to the project plan.
\label{sec:summary}

\section*{Design}%
\label{sec:design}
\begin{figure}[htpb]
  \centering
  \includegraphics[width=0.6\linewidth]{~/Downloads/Screenshot 2022-02-08 at 16-12-04 1TE721 Phase 1.png}
  \caption{The system design layout}%
  \label{fig:p1}
\end{figure}
Figure \ref{fig:p1} shows a brief sketch of the implementation, where in Phase 1, it is not yet connected to the cloud. 

\subsection*{Communication protocols}%
\label{sub:communication_protocols}
1-Wire is used as the bus communication system between the sensor and MCU, which is driven by one of the pins. A command example is sending a reset pulse (falling edge followed by a low signal for 480 microseconds) which can be followed by sending a hexadecimal byte (\verb|0x00| to \verb|0xFF|) that can be decoded as an instruction by the sensor.

The Arduino Nano board has a USB to serial chip which allows the Raspberry Pi to send and receive data using a USB Mini-B cable. This is possible through UART (Universal Asynchronous Receiver/Transmitter) and can be configured according to the datasheet by activating the correct registers and setting the transfer speed for the bytes communicated by the two units.
\section*{Hardware and development tools}%
\label{sec:hardware_and_development_tools}
\begin{enumerate}
  \item Arduino Nano Board (based on an ATmega328p) on a breadboard \cite{atmega328p}
  \item Maxim Integrated DS18B20 1-wire temperature sensor with a 4.7 k$\Omega$ pull-up resistor \cite{interfaceDS18B20}
  \item Raspberry Pi 3B v1.2 running 64 bit Arch Linux ARM \cite{alarmv8}
  \item The Embedded-C software is developed in Neovim, using the CCLS language server and compiled using the AVR-GCC toolchain through GNU Make
\end{enumerate}
\section*{Conclusion}%
\label{sec:conclusion}

Phase 2 is ready to be implemented. The results in Phase 1 produces live data and can be used in Phase 2. If more data is needed, it is easy to generate dummy-data in order to test tools for time-series analysis in the server.

\newpage
\printbibliography
\end{document}

% UART
% Maxim Integr data sheet
% Arch Linux Arm
% Neovim repo
